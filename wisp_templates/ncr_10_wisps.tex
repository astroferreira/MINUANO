\documentclass[12pt]{article}
%
% Overrule the default header size and margins for even and odd pages
%
\usepackage[head=22pt,margin=70pt]{geometry}
\usepackage{datetime}
\newdateformat{mydate}{\THEYEAR~ \monthname[\THEMONTH] \THEDAY}
%%%
%%% Allow including postscript figures
%
\usepackage{epsfig}
\usepackage{graphicx}
\usepackage[english]{babel}
\usepackage{graphicx}
\usepackage{amsmath,amssymb}

\begin{document}
\Large
{\centerline{\bf{NIRCam Wisp Templates}}}
\large
{\centerline{Christopher N. A. Willmer}}
{\centerline{Steward Observatory, University of Arizona}}
{\centerline{\today}}

\section{Introduction}
During the acquisition of NIRCam flat field images using the zodiacal light
background, two types of stray light were noticed.
One are the ``claws'', which are caused by light from very
bright stars (\textsl{K} $\lesssim$ 2) that hits the NIRCam short
wavelength channel spacers coming through the Aft Optics Subsystem. This can
be mitigated by removing these stars from the optical path to the
NIRCam spacers by small changes in the telescope orientation.
``Wisps'' are another type of contamination due to stray light reflected by 
the secondary mirror support system that  shows fairly stable behaviour though
not necessarily present in all images with significant S/N. During the
commissioning activities the IDT+IDS teams found out that this
contamination can be subtracted out prior to the production of the
calibrated images. 

\section{Data and Procedure}
The data used to construct the wisp templates come from
the NIRCam Commissioning Activity \textsl{NRC-10} for proposal 1063
(PI: B. Sunnquist). The Level 1a 
images were copied from MAST and processed using the \textsl{NCDHAS}
reduction code developed by the UofA instrument team (Misselt 2007;
2022) using ground-based calibration files. The $rate$ files (in DN/s)
produced by \textsl{NCDHAS} are identical to the level 2 products coming
from STScI DMS stage 1. The reduction included the removal
removing instrumental signatures (IPC, bias, linearity correction) and
cosmic-ray removal.

All $rate$ files were visually inspected to remove images containing
claws. The remaining images were stacked (by filter
and detector) using the outlier-resistant bi-weight average (Beers et
al. 1990, AJ, 100,32) and  the median-sky image calculated from the division
of the stacked image by the corresponding pixel-flat image.
The last steps in preparing the wisp templates are calculating the
average sky background counts of the median sky (again using the
bi-weight estimator) and subtracting the latter from the median-sky image.

\section{Results}

Wisps are detected at a significant level for three detectors - A3 on
module A and B3 and B4 on Module B and are most prominent in the
F150W2 and F200W filters. Figures 1 and 2 show mosaics of all short
wavelength detectors for three filters (F150W2 in Figure 1, F200W in
Figure 2 and F210M in Figure 3). Two examples of wisp removal are shown
in Figures 4 (for F150W2) and 5 for(F200W) where the left panel shows
the rate file the right panel the wisp-subtracted image.

\begin{figure}
\includegraphics[scale=0.35]{/home/cnaw/Pictures/nrc-10/skyflat_F150W2.jpeg}
  \caption{Example of wisps measured during JWST commissioning in
    filter F150W2 in a mosaic with all short wave detectors. The top row
    of detectors is [A2, A4, B3, B1], bottom row [A1, A3, B4, B2]. The
    wisps are most intense in detectors A3, B3 and B4.}
\end{figure}

\begin{figure}
  \includegraphics[scale=0.35]{/home/cnaw/Pictures/nrc-10/skyflat_F200W.jpeg}
  \caption{Example of wisps observed in the F200W filter; as in the
    previous figure, the top row of detectors is [A2, A4, B3, B1] and the
    bottom row [A1, A3, B4, B2]} 
\end{figure}
\begin{figure}
  \includegraphics[scale=0.35]{/home/cnaw/Pictures/nrc-10/skyflat_F210M.jpeg}
  \caption{Example of wisps observed in F210M, using the same detector
  arrangement of the previous figures.}
\end{figure}
\begin{figure}
  \centering
  \includegraphics[scale=0.53]{jw01063117002_02101_00010_nrcb4_wips.png}
  \caption{Image before (left panel) and after subtracting
    the wisp template. This is commissioning observation
    jw01063117002\_02101\_00001\_nrcb4 using the F150W2 filter. In
    both panels the scale and limits for both panels were locked to be
    identical. The magenta points flag pixels presenting ``NaN''
    values, which are not used in the data processing.}
\end{figure}
\begin{figure}
\centering
  \includegraphics[scale=0.55]{jw01063001003_02101_00001_nrcb4_wisp.png}
  \caption{Image before (left panel) and after (right
    panel) subtracting the wisp template. This is commissioning observation
    jw01063001003\_02101\_00001 using the F200W filter. As in the
    previous figure the scale and limits for both panels were locked to be
    identical.} 
\end{figure}
%\maketitle
%\thispagestyle{empty}
\end{document}
